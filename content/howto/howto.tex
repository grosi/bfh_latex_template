%%%%%%%%%%%%%%%%%%%%%%%%%%%%%%%%%%%%%%%%%%%%%%%%%%%%%%%%%%%%%%%%%%%%%%%%%%%%%%%
% Titel:   Howto
% Autor:   Simon grossenbacher
% Datum:   27.09.2013
% Version: 1.0.0
%%%%%%%%%%%%%%%%%%%%%%%%%%%%%%%%%%%%%%%%%%%%%%%%%%%%%%%%%%%%%%%%%%%%%%%%%%%%%%%
%
%:::Change-Log:::
% Versionierung erfolgt auf folgende Gegebenheiten: -1. Release Versionen
%                                                   -2. Neue Kapitel
%                                                   -3. Fehlerkorrekturen
%
% 0.0.0       Erstellung der Datei
%%%%%%%%%%%%%%%%%%%%%%%%%%%%%%%%%%%%%%%%%%%%%%%%%%%%%%%%%%%%%%%%%%%%%%%%%%%%%%%    
\chapter{Howto}\label{ch:howto}
    Eine kurze Installationsanleitung zu \LaTeX unter Windows.
    %
    %
    %
    \section{Miktex}\label{s:miktex}
        Miktex ist eine Tex-Distribution f�r Windows.
        %
        %
        \subsection{Installation}
            \begin{enumerate}
                \item Den passenden Net-Installer von der Projekt-Homepage herunterladen: \url{http://miktex.org/download}
                \image{content/howto/image/miktex_1}{scale=.35}{htbp}[Installations-Datei w�hlen]
                %
                \item Installer starten und als Installationsoption \textsf{MikTex Downloading}, sowie \textsf{Complete} angeben.\footnote{Achtung: Kann �ber 1GB Speicher ben�tigen} Anschliessend werden die verschiedenen \LaTeX -Pakete heruntergeladen.
                %
                \item Nach dem Download den Installer nochmals starten und dieses Mal \textsl{Install MikTex} anw�hlen. MikTex wird nun installiert.
            \end{enumerate}
    %
    %
    %TexMaker
    \section{TexMaker}\label{s:texmaker}
        Optionen $\longrightarrow$ Texmaker konfigurieren
        \image{content/howto/image/texmaker_1}{scale=0.35}{htbp}[Konfiguration Tab 1]
        \image{content/howto/image/texmaker_2}{scale=0.35}{htbp}[Konfiguration Tab 2]
    %
    %
    %Hinweise
    \section{Hinweise}
        \begin{itemize}
            \item Die \texttt{report.tex} Datei unter Optionen zur Masterdatei machen
            %
            \item \textbf{WICHTIG:} Immer darauf achten, dass die \texttt{*.tex} Dateien in ISO-8859 Kodiert und mit Windows Zeilenenden gespeichert werden.
            \image{content/howto/image/hinweise}{scale=.35}{htpb}[Speichern unter... Dialog]
            %
            \item An Apple/Linux Benutzer: Keine \texttt{*.eps} Bilder verwenden! Windows unterst�tzt dieses Format standardm�ssig nicht! 
        \end{itemize}